\documentclass[letterpaper]{article}
%======================================================================
% LOCAL DOCUMENT CONFIG

%% LaTeX Packages ------------
\usepackage[utf8]{inputenc}
\usepackage[T1]{fontenc}
\usepackage{textcomp}

\usepackage{setspace}
\usepackage{calc}

\usepackage{amsmath,amssymb,mathrsfs}

%% page style ----------------
\usepackage[left=1.0in,right=1.0in,top=1in,bottom=1.25in]{geometry}
\usepackage{fancyhdr}
% modified from lshort.sty
\pagestyle{fancy}
\renewcommand{\sectionmark}[1]{}
\renewcommand{\rightmark}{Kuczenski, Leduc \& Messner --- 2010}
\rhead[\fancyplain{}{\bfseries\thepage}]
    {\fancyplain{}{\bfseries\rightmark}}
\lhead[\fancyplain{}{\bfseries\leftmark}]
    {\fancyplain{}{\bfseries\thepage}}
\cfoot[]{}
\addtolength{\headheight}{15pt}
% ----------------------------


\usepackage{pstricks-add}

\newcommand{\sumnode}[1]{\cnode[fillstyle=solid,fillcolor=white]{0.2cm}{#1}
  \rput(0,0){$+$}}
\newcommand{\nodebox}[5]{%
  \rput(0,0){\rnode{#1}{\psframebox[#2]{\rule{0pt}{#4}\rule{#3}{0pt}}}}
  \rput(0,0){\psframebox[fillstyle=none,linestyle=none]{\parbox{#3 -
        0.2cm}{\centering #5}}}
}

\newcommand{\Ca}{$\left[\textrm{Ca}^{2+}\right]_i$}
\newcommand{\matlab}{\textsc{Matlab}}
\newcommand{\Matlab}{\matlab}
\newcommand{\TM}{$^{\textsc{tm}}$}
\newcommand{\email}[1]{\texttt{\href{mailto:#1}{#1}}}
\newcommand{\doi}[1]{doi:~\href{http://dx.doi.org/#1}{\small\texttt{#1}}}

% hyphenation
\hyphenation{mi-cro-flu-idic mi-cro-flu-idics nano-liter homeo-stasis
  iono-phore iono-phores Massa-chusetts para-dig-matic ana-lytic}

%%
%% MAIN DOCUMENT SETTINGS
%%
\def\figlabel{\large\textsf}

%% FONTS
\usepackage[T1]{fontenc}
\usepackage{fourier}
%\usepackage{mathptmx} % for base font - times
\usepackage[scaled]{helvet} % for sans serif
%\usepackage{fourier}
%\usepackage{mathptmx} % for base font - times
%\usepackage[scaled]{helvet} % for sans serif
%\usepackage[scaled]{uarial} % per RCR requirements: for sans serif - arial


% instead of psstyle.tex:
\newpsstyle{dashed}{linestyle=dashed,dash=7pt 4pt,linewidth=0.6pt,linecap=1}
\newpsstyle{standard}{arrowsize=3pt 2,%
  arrowinset=0.15,arrowlength=1.8,linearc=0.1} % formerly vague
%\newpsstyle{vague}{arrowsize=3pt 2,arrowinset=0.15,arrowlength=1.8,linearc=0.1}
\newpsstyle{grid}{linestyle=dashed,dash=4pt 2pt,linewidth=0.25pt,
  linecolor=gray,showpoints=false}
\newpsstyle{process}{dimen=outer,linestyle=solid,linecolor=darkgray,%
  shadow=false,fillstyle=solid,linewidth=0.85pt}

% for plots
\newrgbcolor{darkblue}{0 0 0.55 }
\newrgbcolor{darkgreen}{0 0.55 0}
\newrgbcolor{steelblue}{0.27 0.51 0.71}

\newpsstyle{vline}{linewidth=0.3pt,linecolor=gray}
\newpsstyle{hline}{linewidth=0.35pt,linecolor=gray,linestyle=dashed,%
  dash=5pt 1pt 1pt 1pt}
\newpsstyle{dim}{linewidth=0.85pt,linecolor=black,tbarsize=6pt 5,offset=12pt}
\newpsstyle{errorbar}{linewidth=0.65pt,linecolor=darkgray,tbarsize=4pt 4,offset=0pt}

% for calcium
\newrgbcolor{fluofourAM}{0.05 0.55 0.1}
%\newrgbcolor{fluofourAM}{0.05 0.15 0.95}

\newpsstyle{decay}{linewidth=2pt,strokeopacity=0.75,linecolor=steelblue}
%\newpsstyle{exposure}{linewidth=1.2pt,linecolor=gray}
\newpsstyle{fluor}{linestyle=none,showpoints=true,dotstyle=*,dotscale=0.85,linecolor=fluofourAM}
\newpsstyle{filt}{linewidth=0.8pt,linecolor=darkgray,showpoints=false}
\newpsstyle{vlim}{linewidth=0.3pt,linecolor=gray}
\newpsstyle{exposure}{linewidth=1.2pt,linecolor=blue}
\newpsstyle{cella}{linecolor=darkblue}
\newpsstyle{cellb}{linecolor=darkgreen,linestyle=dashed,dash=5pt 2pt}








\begin{document}

\begin{figure}
\centering
\input{step-mini.fig}
\caption{Step Response of fibroblasts to ionomycin at the specified
  concentration.  Each panel shows a single experiment with multiple cells
  in the field of view.  Results are typical of 29 experiments including
  148 cells. }
\end{figure}

%% \begin{figure}
%% \centering
%% \input{step-response.fig}
%% \caption{Step Response.}
%% \end{figure}

\begin{figure}
\centering
\input{step-analysis.fig}
\input{step-statistics.fig}
\caption{Step Response Modeling.  All cells were monitored until they
  reached a peak intensity, after which they began their recovery towards
  the baseline intensity.  After the rate of recovery reached a maximum,
  the subsequent recovery was approximated as an exponential decay.  The
  cells were characterized in terms of the time to rise to peak intensity
  and the characteristic time constant of the exponential decay. (A) A
  diagram showing salient figures in the step response analysis.  (B) Rise
  time as a function of ionomycin concentration.  Each datapoint shows mean
  and standard deviation.  The number of cells in each sample is shown in
  parentheses.  For 100 nM, $t_{rise}=99\pm61$~s; for 1 $\mu$M,
  $t_{rise}=26\pm8$~s; for 10 $\mu$M, $t_{rise}=14.5\pm4.0$~s.  (C)
  Exponential decay time constant as a function of ionomycin concentration.
  Results shown as in (B).  For 100 nM, $\tau=64\pm16$~s; for 1 $\mu$M,
  $\tau=28\pm9$~s; for 10 $\mu$M, $\tau=21\pm4.3$~s.}
\end{figure}

\begin{figure}
\centering
\input{cell-schematic.fig}
\caption{Identifying linear characteristics of cellular response to
  ionophore.  Estimates of cellular exposure to ionomycin were compared
  with measurements of cell fluorescence to ascertain whether the cellular
  response can be described by a linear model.  The input was the mean
  concentration of ionomycin over the cell's apical surface versus time.
  The output was the change from baseline levels of the cell's mean
  fluorescent intensity.}
\end{figure}

\begin{figure}
\centering
\input{impulse-response.fig}
\caption{Measuring Impulse Response.  (some data was previously published
  in PLoS, figure 4b.) }
\end{figure}

\begin{figure}
\centering
\input{model-response.fig}
\caption{Pulse Response Identification.  An iterative parameter estimation
  algorithm was used to identify potential models.  A second-order linear
  model was found to fit the data consistently in 23 of 31 trials.  The
  remaining 8 resulted in ill-conditioned models. }
\end{figure}

\begin{figure}
\centering
\input{model-firstorder.fig}
\caption{Pulse Response Modeling.  The pulse responses were found to be
  well-described as two first-order systems working in parallel and acting
  in opposite directions ($n=23$).  The slower system, which dominated the
  response, had a time constant of 31$\pm$14 sec.  The faster system had a
  time constant of roughly 1--6 s and was roughly half as effective at
  influencing fluorescent intensity as the dominant system (relative gain
  $-0.52\pm0.11$).  $K_{sys}$ represents system gain, which is correlated
  to the magnitude of the increase in fluorescence and was
  wide-ranging (from 1.17 to 70.2).}\label{fig:model}
\end{figure}

\begin{figure}
\centering
\input{pulse-poles.fig}
\caption{Pulse response time constants as a function of cell
  responsiveness.  Horizontal axis is the system gain computed for each
  trial (see Figure~\ref{fig:model} above).  Vertical axis shows the time
  constants of the two first-order systems.  These are the numerical
  inverses of the eigenvalues of the characteristic equation of the linear
  model. }
\end{figure}


%% \begin{figure}[t]
%% \centering
%% \input{firstorder.fig}
%% %\caption{First Model -- Lead network.}
%% %\end{figure}
%% \caption{Modeling calcium response as two first-order systems.}
%% \end{figure}


%% \begin{figure}
%% \centering
%% \input{zp.fig}
%% \caption{Eigenvalues.}
%% \end{figure}

%% \begin{figure}
%% \centering
%% \input{lead-network.fig}
%% \caption{First Model: Lag network.}
%% \end{figure}

\end{document}


% cell1 - 0.1765
% cell4 - 0.4118
